\documentclass[12pt, a4paper]{article}
\usepackage[utf8]{inputenc}
\usepackage[T1]{fontenc}
\usepackage[french]{babel} % French language support
\usepackage{graphicx}
\usepackage{geometry}
\usepackage{listings}
\usepackage{xcolor}
\usepackage{hyperref}
\usepackage{float}     % To force image positioning
\usepackage{fancyhdr}  % For professional headers/footers
\usepackage{caption}

% Page Geometry
\geometry{left=2.5cm, right=2.5cm, top=2.5cm, bottom=2.5cm}

% Header and Footer Setup
\pagestyle{fancy}
\fancyhf{}
\rhead{\textbf{TP Laravel - Gestion Étudiants}}
\lhead{ENSAT}
\cfoot{\thepage}

% Code Highlighting Colors
\definecolor{codegray}{rgb}{0.5,0.5,0.5}
\definecolor{codepurple}{rgb}{0.58,0,0.82}
\definecolor{backcolour}{rgb}{0.95,0.95,0.92}
\definecolor{laravelred}{rgb}{0.96, 0.29, 0.23}

% Code Style Configuration
\lstdefinestyle{mystyle}{
    backgroundcolor=\color{backcolour},   
    commentstyle=\color{codegray},
    keywordstyle=\color{laravelred}\bfseries,
    numberstyle=\tiny\color{codegray},
    stringstyle=\color{codepurple},
    basicstyle=\ttfamily\footnotesize,
    breakatwhitespace=false,         
    breaklines=true,                 
    captionpos=b,                    
    keepspaces=true,                 
    numbers=left,                    
    numbersep=5pt,                  
    showspaces=false,                
    showstringspaces=false,
    showtabs=false,                  
    tabsize=4,
    frame=single
}

\lstset{style=mystyle}

\begin{document}

%----------------------------------------------------------------------------------------
%	TITLE PAGE
%----------------------------------------------------------------------------------------
\begin{titlepage}
    \centering
    % Placeholder for Logo
    % \includegraphics[width=0.3\textwidth]{logo_ensat.png}\\[1cm] 
    
    {\scshape\LARGE École Nationale des Sciences Appliquées de Tanger \par}
    \vspace{1cm}
    {\scshape\Large Rapport de Travaux Pratiques \par}
    \vspace{1.5cm}
    
    {\huge\bfseries Application de Gestion des Étudiants\par}
    \vspace{0.5cm}
    {\large Basée sur Laravel 9 et Authentification Firebase \par}
    
    \vspace{2cm}
    
    \begin{minipage}{0.4\textwidth}
        \begin{flushleft} \large
        \emph{Réalisé par :}\\
        [VOTRE NOM]
        \end{flushleft}
    \end{minipage}
    \begin{minipage}{0.4\textwidth}
        \begin{flushright} \large
        \emph{Encadré par :}\\
        Pr. Khalid AMECHNOUE
        \end{flushright}
    \end{minipage}
    
    \vfill
    
    {\large \today}
\end{titlepage}

%----------------------------------------------------------------------------------------
%	CONTENT
%----------------------------------------------------------------------------------------

\tableofcontents
\newpage

\section{Introduction}
Dans le cadre de notre formation en ingénierie informatique à l'ENSAT, il nous a été demandé de réaliser une application web robuste pour la gestion des étudiants. L'objectif principal est de mettre en œuvre une architecture MVC moderne utilisant le framework \textbf{Laravel 9}.

L'application doit répondre à trois besoins fonctionnels majeurs :
\begin{itemize}
    \item Une authentification sécurisée différenciant les rôles (Admin/Student).
    \item \textbf{Nouveau :} Une connexion sociale via Google (Firebase Auth).
    \item Une gestion complète (CRUD) des étudiants pour les administrateurs.
\end{itemize}

Ce rapport détaille les choix techniques, la modélisation de la base de données, ainsi que l'implémentation des fonctionnalités clés, incluant l'intégration de l'API Google Sign In.

\section{Environnement Technique}
Le projet a été développé en utilisant la stack technologique suivante, adaptée aux contraintes de l'environnement (PHP 8.0) :

\begin{itemize}
    \item \textbf{Framework}: Laravel 9.x
    \item \textbf{Langage}: PHP 8.0
    \item \textbf{Base de données}: MySQL (XAMPP)
    \item \textbf{Authentification Externe}: Firebase Auth (Google Cloud Platform)
    \item \textbf{Frontend}: Blade Templates + TailwindCSS (Laravel Breeze)
\end{itemize}

\section{Conception et Modélisation}

\subsection{Schéma de la Base de Données}
La base de données \texttt{ensat\_db} repose sur une relation One-to-One entre la table d'authentification standard \texttt{users} et la table métier \texttt{students}.

\subsection{Structure des Tables}
\subsubsection{Table \texttt{users}}
Nous avons enrichi la migration par défaut pour inclure la gestion des rôles :
\begin{lstlisting}[language=PHP]
Schema::create('users', function (Blueprint $table) {
    $table->id();
    $table->string('name');
    $table->string('email')->unique();
    // Distinction Admin vs Student
    $table->enum('role', ['admin', 'student'])->default('student'); 
    $table->string('password');
    $table->timestamps();
});
\end{lstlisting}

\subsubsection{Table \texttt{students}}
Cette table contient les données académiques et est liée à l'utilisateur par une clé étrangère.
\begin{lstlisting}[language=PHP]
Schema::create('students', function (Blueprint $table) {
    $table->id();
    $table->foreignId('user_id')->constrained()->onDelete('cascade');
    $table->string('cne')->unique(); // Code National Étudiant
    $table->string('sector');        // Filière (ex: GINF)
    $table->string('city');          // Ville d'origine
    $table->timestamps();
});
\end{lstlisting}

\newpage
\section{Implémentation des Fonctionnalités Core}

\subsection{Sécurité et Middleware}
Pour sécuriser l'application, nous avons mis en place un middleware personnalisé \texttt{RoleMiddleware}. Il intercepte les requêtes HTTP pour vérifier si l'utilisateur possède les droits nécessaires.

\begin{lstlisting}[language=PHP]
// App/Http/Middleware/RoleMiddleware.php
public function handle(Request $request, Closure $next, $role)
{
    if (! $request->user() || $request->user()->role !== $role) {
        abort(403, 'Accès interdit aux non-administrateurs.');
    }
    return $next($request);
}
\end{lstlisting}

\subsection{Module Administrateur (CRUD)}
L'administrateur dispose d'un contrôle total. Lors de la création d'un étudiant, nous utilisons Eloquent pour assurer l'intégrité des données (création simultanée du compte de connexion et du profil étudiant).

\section{Authentification Firebase (Google Sign In)}

Une évolution majeure du projet a été l'intégration de \textbf{OAuth 2.0} via Firebase pour permettre la connexion avec un compte Google.

\subsection{Architecture de l'Authentification}
Le processus se déroule en trois étapes :
\begin{enumerate}
    \item \textbf{Frontend (JS)} : L'utilisateur se connecte via la popup Google gérée par le SDK Firebase. Firebase renvoie un \texttt{id\_token} sécurisé.
    \item \textbf{Transmission} : Ce token est envoyé via une requête POST AJAX vers notre backend Laravel.
    \item \textbf{Backend (Laravel)} : Le contrôleur vérifie la signature du token via les clés publiques de Google, récupère l'email, et connecte l'utilisateur localement.
\end{enumerate}

\subsection{Implémentation Backend (\texttt{FirebaseController})}
Nous avons utilisé la librairie \texttt{firebase/php-jwt} pour valider les tokens JWT.

\begin{lstlisting}[language=PHP]
// App/Http/Controllers/Auth/FirebaseController.php

public function loginWithGoogle(Request $request)
{
    $token = $request->input('id_token');
    
    // 1. Récupération des clés publiques Google
    $keysJson = file_get_contents('https://www.googleapis.com/...');
    $keys = json_decode($keysJson, true);
    
    // 2. Décodage et Validation du Token
    $decoded = JWT::decode($token, $this->parseKeys($keys));
    $email = $decoded->email;

    // 3. Connexion ou Création Automatique (Auto-Register)
    $user = User::where('email', $email)->first();

    if (!$user) {
        $user = User::create([
            'name' => $decoded->name,
            'email' => $email,
            'password' => Hash::make(uniqid()), // Mdp aléatoire
            'role' => 'student',
        ]);
        // Création du profil étudiant vide associé
        Student::create(['user_id' => $user->id, ...]);
    }
    
    // 4. Login Session Laravel
    Auth::login($user);
    
    return response()->json(['success' => true]);
}
\end{lstlisting}

\subsection{Configuration Frontend}
Le fichier \texttt{login.blade.php} a été modifié pour inclure le SDK Firebase Modular (v9+).
\begin{lstlisting}[language=HTML]
<script type="module">
  import { initializeApp } from "https://www.gstatic.com/firebasejs/9.22.0/firebase-app.js";
  import { getAuth, signInWithPopup, GoogleAuthProvider } ...

  const firebaseConfig = {
      apiKey: "AIzaSyCLNnsYl5Df...",
      authDomain: "tpauth-7cc38.firebaseapp.com",
      projectId: "tpauth-7cc38",
      // ...
  };
  
  // Gestion du clic sur "Sign in with Google"
  signInWithPopup(auth, provider).then((result) => {
      // Envoi du token au backend Laravel
  });
</script>
\end{lstlisting}

\newpage
\section{Conclusion}
Ce projet nous a permis de maîtriser les fondamentaux de \textbf{Laravel 9} et d'explorer l'intégration de services tiers comme \textbf{Firebase}. L'application offre désormais une double méthode d'authentification (Classique + Google), garantissant flexibilité et sécurité.

Perspectives d'amélioration :
\begin{itemize}
    \item Synchroniser la photo de profil Google avec l'application.
    \item Gérer les erreurs Firebase (ex: popup fermée) plus finement côté UI.
\end{itemize}

\end{document}
